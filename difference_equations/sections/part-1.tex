\section{Difference Equation Version}
\label{sec:difference-equation-version}

We work with the following coupled difference equations:

\begin{equation}
    \label{equation:difference:m}
    M_{t + 1} = M_t + r M_t \bigg[ 1 - \frac{M_t}{K} \bigg] - c \cdot M_t S_t
\end{equation}

\begin{equation}
    \label{equation:difference:s}
    S_{t + 1} = S_t + \epsilon \cdot c \cdot M_t S_t - d \cdot S_t^{2}
\end{equation}

The parameters in Equations~\ref{equation:difference:m} and~\ref{equation:difference:s} are:

\begin{itemize}
    \item $M$ is the population of \textit{menhaden},
    \item $S$ is the population of \textit{stripers},
    \item $t$ is \textit{time},
    \item $r$ is the \textit{reproduction rate} of menhaden,
    \item $K$ is the \textit{carrying capacity} of menhaden,
    \item $c$ is the \text{consumption rate} of menhaden by stripers,
    \item $\epsilon$ is the \textit{efficiency} of consumption by stripers, and
    \item $d$ is the \textit{death rate} of stripers.
\end{itemize}


\subsection*{a.}

\begin{table}[h!]
    \begin{center}
        \begin{tabular}{c|c|c}
            \textbf{Parameter} & \textbf{Unit} & \textbf{Value} \\
            \hline
            $r$ & -- & 0.50 \\
            $K$ & $M$ & 100 \\
            $c$ & $S^{-1}$ & 0.05 \\
            $\epsilon$ & $S \cdot M^{-1}$ & 0.15 \\
            $d$ & $S^{-1}$ & 0.05 \\
        \end{tabular}
        \caption{Difference Equations Parameter Values}
        \label{table:difference-parameters}
    \end{center}
\end{table}


\newpage
\subsection*{b.}

When in equilibrium, the populations no longer change.
Therefore, to solve for the isoclines, we set:
\begin{itemize}
    \item $M_{t + 1} = M_t = M^{\star}$, and
    \item $S_{t + 1} = S_t = S^{\star}$.
\end{itemize}

The equations for the isoclines are:

\begin{equation}
    \label{equation:difference:m-isocline}
    S^{\star} = \frac{r}{s} \bigg( 1 - \frac{M^{\star}}{K} \bigg)
\end{equation}

\begin{equation}
    \label{equation:difference:s-isocline}
    S^{\star} = \frac{\epsilon \cdot c}{d} M^{\star}
\end{equation}


\subsection*{c.}

The joint equilibrium is $M^{\star} = 40$, and $S^{\star} = 6$.


\subsection*{d.}

The chosen values for all parameters are shown in Table~\ref{table:difference-parameters}.


\subsection*{e. and f.}

Figure~\ref{figure:difference-single} shows how the populations change when starting with $3$ pogies and $1$ striper.

\begin{figure}[ht!]
    \centering
    \includegraphics[width=6in]{images/difference-plot-single.png}
    \caption{Difference Equations and corresponding Isoclines.}
    \label{figure:difference-single}
\end{figure}

Figure~\ref{figure:difference-multiple} shows how the populations change when starting with several different initial populations of pogies and stripers.

\begin{figure}[ht!]
    \centering
    \includegraphics[width=6in]{images/difference-plot-multiple.png}
    \caption{Difference Equations and corresponding Isoclines.}
    \label{figure:difference-multiple}
\end{figure}

Figures~\ref{figure:difference-single} and~\ref{figure:difference-multiple} both show that the populations converge on the equilibrium we calculated earlier.


\subsection*{g.}

Halving the reproduction rate of menhaden to $0.25$ should decrease the equilibrium populations of both fish.
The curves converge on $(25.1, 3.7)$, as expected.

Doubling the consumption rate to $0.10$ should vastly decrease the population of menhaden, and somewhat decrease the population of stripers.
The curves converge on $(14.3, 4.3)$, as can be seen in Figure~\ref{figure:difference-double-c}.

\begin{figure}[ht!]
    \centering
    \includegraphics[width=6in]{images/difference-double-c.png}
    \caption{Doubling the Consumption Rate.}
    \label{figure:difference-double-c}
\end{figure}

Doubling the ecological efficiency to $0.30$ should increase the population of stripers which would cause a decrease in the population of pogies.
We see this happen with the populations converging to $(25, 7.5)$.

Doubling the death rate of strieprs to $0.10$ should decrease the population of stripers and increase the population of pogies.
We see this happen with the populations converging to $(57.1, 4.3)$ in Figure~\ref{figure:difference-double-d}.

\begin{figure}[ht!]
    \centering
    \includegraphics[width=6in]{images/difference-double-d.png}
    \caption{Doubling the Death Rate.}
    \label{figure:difference-double-d}
\end{figure}

In general, a steeper isocline corresponds to larger changes in the corresponding population with each time step.
