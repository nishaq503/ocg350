\section{Differential Equation Version}
\label{sec:defferential-equation-version}

We work with the following coupled difference equations:

\begin{equation}
    \label{equation:differential:m}
    \frac{\delta M}{\delta t} = M_t + r M_t \bigg[ 1 - \frac{M_t}{K} \bigg] - c \cdot M_t S_t
\end{equation}

\begin{equation}
    \label{equation:differential:s}
    \frac{\delta S}{\delta t} = S_t + \epsilon \cdot c \cdot M_t S_t - d \cdot S_t^{2}
\end{equation}

The parameters in Equations~\ref{equation:differential:m} and~\ref{equation:differential:s} are:

\begin{itemize}
    \item $\frac{\delta M}{\delta t}$ is the \textit{change} population of menhaden per unit time,
    \item $\frac{\delta S}{\delta t}$ is the \textit{change} population of stripers per unit time, and
    \item the rest of the parameters are the same as before.
\end{itemize}


\subsection*{a.}

\begin{table}[h!]
    \begin{center}
        \begin{tabular}{c|c|c}
            \textbf{Parameter} & \textbf{Unit} & \textbf{Value} \\
            \hline
            $r$ & -- & 0.50 \\
            $K$ & $M$ & 100 \\
            $c$ & $S^{-1}$ & 0.05 \\
            $\epsilon$ & $S \cdot M^{-1}$ & 0.15 \\
            $d$ & $S^{-1}$ & 0.05 \\
        \end{tabular}
        \caption{Differential Equations Parameter Values}
        \label{table:differential-parameters}
    \end{center}
\end{table}


\subsection*{b.}

At the equilibrium, the populations no longer change.
Therefore, to solve for the isoclines, we set:
\begin{itemize}
    \item $\frac{\delta M}{\delta t} = 0$ and $M^{\star} = M_t$, and
    \item $\frac{\delta S}{\delta t} = 0$ and $S^{\star} = S_t$.
\end{itemize}

The equations for the isoclines turned out to be the same as in Equations~\ref{equation:difference:m-isocline} and~\ref{equation:difference:s-isocline}.


\subsection*{c.}

The joint equilibrium is $M^{\star} = 40$, and $S^{\star} = 6$.


\subsection*{d.}

The chosen values for all parameters are shown in Table~\ref{table:differential-parameters}.


\subsection*{e.}

The time-series plot of the populations of the fish is shown in Figure~\ref{figure:differential-time-series-single}.

\begin{figure}[ht!]
    \centering
    \includegraphics[width=6in]{images/differential-time-series-single.png}
    \caption{Time Series of Populations}
    \label{figure:differential-time-series-single}
\end{figure}


\subsection*{f.}

Figure~\ref{figure:differential-single} shows how the populations change when starting with $3$ pogies and $1$ striper.

\begin{figure}[ht!]
    \centering
    \includegraphics[width=6in]{images/differential-plot-single.png}
    \caption{Differential Equations and corresponding Isoclines.}
    \label{figure:differential-single}
\end{figure}

Figure~\ref{figure:differential-multiple} shows how the populations change when starting with several different initial populations of pogies and stripers.

\begin{figure}[ht!]
    \centering
    \includegraphics[width=6in]{images/differential-plot-multiple.png}
    \caption{Differential Equations and corresponding Isoclines.}
    \label{figure:differential-multiple}
\end{figure}

Figures~\ref{figure:differential-single} and~\ref{figure:differential-multiple} both show that the populations converge on the equilibrium we calculated earlier.


\subsection*{g.}

Halving the reproduction rate of menhaden to $0.25$ should decrease the equilibrium populations of both fish.
The curves converge on $(25, 3.8)$, as expected.
This is very slightly different from our difference-equations model.

Doubling the consumption rate to $0.10$ should vastly decrease the population of menhaden, and somewhat decrease the population of stripers.
The curves converge on $(14.3, 4.3)$.
This is identical to our difference-equations model.

Doubling the ecological efficiency to $0.30$ should increase the population of stripers which would cause a decrease in the population of pogies.
We see this happen with the populations converging to $(25, 7.5)$.
This is also identical to our difference-equations model.

Doubling the death rate of strieprs to $0.10$ should decrease the population of stripers and increase the population of pogies.
We see this happen with the populations converging to $(57.1, 4.3)$.
This is also identical to our difference-equations model.
